\documentclass{beamer}

\mode<presentation>
{
  \usetheme{Singapore}
  \usecolortheme{rose}
  \setbeamercovered{transparent}
}

\usepackage[english]{babel}
\usepackage[latin1]{inputenc}
\usepackage{times}
\usepackage{listings}
\usepackage[T1]{fontenc} 
% Or whatever. Note that the encoding and the font should match. If T1
% does not look nice, try deleting the line with the fontenc.
\usepackage{amsmath}

\newcommand{\linespace}{\vskip 0.25cm}

\definecolor{MyForestGreen}{rgb}{0,0.7,0} 
\newcommand{\tableemph}[1]{{#1}}
\newcommand{\tablewin}[1]{\tableemph{#1}}
\newcommand{\tablemid}[1]{\tableemph{#1}}
\newcommand{\tablelose}[1]{\tableemph{#1}}

\definecolor{MyLightGray}{rgb}{0.6,0.6,0.6}
\newcommand{\tabletie}[1]{\color{MyLightGray} {#1}}

\lstset{language=Java}

% The text in square brackets is the short version of your title and will be used in the
% header/footer depending on your theme.
\title[Developmental plasticity in N-gram GP]{Thread Scheduler Efficiency Improvements \\ for Multiprocessor Multicore Systems [Draft]}

% Sub-titles are optional - uncomment and edit the next line if you want one.
% \subtitle{Why does sub-tree crossover work?} 

% The text in square brackets is the short version of your name(s) and will be used in the
% header/footer depending on your theme.
\author[DFrz]{Daniel Collin Frazier}

% The text in square brackets is the short version of your institution and will be used in the
% header/footer depending on your theme.
\institute[U of Minn, Morris]
{
  Division of Science and Mathematics \\
  University of Minnesota, Morris \\
  Morris, Minnesota, USA
}

% The text in square brackets is the short version of the date if you need that.
\date[November '17, UMM, Minnesota] % (optional)
{18 November 2017 \\ UMM, Minnesota}

% Delete this, if you do not want the table of contents to pop up at
% the beginning of each subsection:
\AtBeginSection[]
{
  \begin{frame}<beamer>
    \frametitle{Outline}
    \tableofcontents[currentsection, hideothersubsections]
  \end{frame}
}

\begin{document}

\begin{frame}
  \titlepage
\end{frame}

% For a 20-25 minute senior seminar talk you probably want something like:
% - Two or three major sections (other than the summary).
% - At *most* three subsections per section.
% - Talk about 30s to 2min per frame. So there should probably be between
%   15 and 30 frames, all told.

\section*{Overview}

\subsection*{Introduction}

\begin{frame}
  \frametitle{Introduction}
  
  \begin{columns}
  \begin{column}{0.6\textwidth}
  \begin{itemize}
  	\item Introduce Topic and where the talk is going.
  \end{itemize}
  \end{column}
  \begin{column}{0.4\textwidth}
  \end{column}
  \end{columns}
\end{frame}

\subsection*{Outline}

\begin{frame}
  \frametitle{Outline}
  \tableofcontents[hideallsubsections]
\end{frame}

\section[Background]{Background}

\subsection[Threads]{Threads and Multithreading}

\begin{frame}
\frametitle{What are Threads?}

\begin{itemize}
  \item Added developmental plasticity to N-gram GP using Incremental Fitness-based Development (IFD).
\end{itemize}

\end{frame}

\begin{frame}
\frametitle{Example Multithreaded Program}

\begin{lstlisting}

        public String getStringFromGUI()
        {
            JFrame window = new JFrame("Enter text and press Enter");
            Container container = window.getContentPane();

            JTextField textbox = new JTextField();
            textbox.addActionListener(new ActionListener() {
                @Override
                public void actionPerformed(ActionEvent e) {
                    //Ran asynchronously in another thread
                    self.notify();
                    window.dispatchEvent(new WindowEvent(window, WindowEvent.WINDOW_CLOSING));
                    window.setVisible(false);
                }
            });

            container.add(textbox);
            window.pack();

            window.setVisible(true);
            wait();
            return textbox.getText();
        }
\end{lstlisting}

\end{frame}

\subsection[parallel]{Parallel Programming}


\section[Conclusions]{Conclusions}

\begin{frame}
\frametitle{Conclusions}

\begin{itemize}
  \item Added developmental plasticity to N-gram GP using Incremental Fitness-based Development (IFD).
\end{itemize}

\begin{itemize}
  \item IFD consistently improved N-gram GP performance on suite of test problems.
  
  \linespace
  
  \item ``Knocking out'' IFD shows it's valuable in all phases, even if it wasn't used earlier in a run.

  \linespace
  
  \item IFD generates more complex, less converged probability tables.
  \item IFD generates more modules/loops \& uses more low-probability paths.
\end{itemize}

\begin{itemize}
  \item Currently exploring applications to dynamic environments.
\end{itemize}

\end{frame}

\begin{frame}
	\frametitle{Thanks!}
	
	Thank you for your time and attention!
		
	\linespace
	\linespace
	
	Contact:  
	\begin{itemize}
		\item \texttt{mcphee@morris.umn.edu}
		\item \url{http://www.morris.umn.edu/~mcphee/}
	\end{itemize}
	
	\linespace
	\linespace
	
	\begin{center}
	{\huge Questions?}
	\end{center}
\end{frame}

\section*{References}

\begin{frame} 
	\frametitle{References} 
	
	\begin{thebibliography}{lskdjf}
	
	\bibitem{McPhee:2009:gecco}
N.~F. McPhee, E.~Crane, S.~Lahr, and R.~Poli.
\newblock Developmental Plasticity in Linear Genetic Programming.
\newblock In G\"unther Raidl, \emph{et al}, editors, {\em GECCO '09}, pages 1019--1026, Montr\'eal, Qu\'ebec, Canada, 2009.
	
	\bibitem{citeulike:3452411}
	R.~Poli and N.~McPhee.
\newblock A linear estimation-of-distribution {GP} system.
\newblock In M.~O'Neill, \emph{et al}, editors, {\em EuroGP 2008}, volume
  4971 of {\em LNCS}, pages 206--217, Naples,
  26-28 Mar. 2008. Springer.
  
  	\end{thebibliography}
	
	\linespace
	\begin{center}
	See the GECCO '09 paper for additional references.
	\end{center}
\end{frame} 

\end{document}


