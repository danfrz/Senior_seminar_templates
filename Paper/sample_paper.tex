% This is a sample document using the University of Minnesota, Morris, Computer Science
% Senior Seminar modification of the ACM sig-alternate style. Much of this content is taken
% directly from the ACM sample document illustrating the use of the sig-alternate class. Certain
% parts that we never use have been removed to simplify the example, and a few additional
% components have been added.

% See https://github.com/UMM-CSci/Senior_seminar_templates for more info and to make
% suggestions and corrections.

\documentclass{sig-alternate}
\usepackage{color}
\usepackage[colorinlistoftodos]{todonotes}

%%%%% Uncomment the following line and comment out the previous one
%%%%% to remove all comments
%%%%% NOTE: comments still occupy a line even if invisible;
%%%%% Don't write them as a separate paragraph
%\newcommand{\mycomment}[1]{}

\begin{document}

% --- Author Metadata here ---
%%% REMEMBER TO CHANGE THE SEMESTER AND YEAR AS NEEDED
\conferenceinfo{UMM CSci Senior Seminar Conference, December 2017}{Morris, MN}

\title{Thread Scheduler Efficiency Improvements for Multiprocessor Multicore Non-Uniform Memory Access Systems}

\numberofauthors{1}

\author{
% The command \alignauthor (no curly braces needed) should
% precede each author name, affiliation/snail-mail address and
% e-mail address. Additionally, tag each line of
% affiliation/address with \affaddr, and tag the
% e-mail address with \email.
\alignauthor
Daniel C. Frazier\\
	\affaddr{Division of Science and Mathematics}\\
	\affaddr{University of Minnesota, Morris}\\
	\affaddr{Morris, Minnesota, USA 56267}\\
	\email{frazi177@morris.umn.edu}
}

\maketitle
\begin{abstract}
[Abstract contents]
\end{abstract}

\keywords{Scheduling, Threads, Multiprocessing, Processors}

\section{Introduction}
\label{sec:intro}

[Introduction contents]

\section{Background}
\label{sec:bg}

[Body text]

\subsection{Threads and Threading}
\label{sec:threads}

[Body text]

\subsection{Completely Fair Scheduler}
\label{sec:cfs}

[Body text]

\subsection{Cache on NUMA Systems}
\label{sec:cache}

[Body text]

\subsection{Cache Locality}
\label{sec:cachelocality}

[Body text]

\section{Methods}
\label{sec:methods}

[Body text]

\subsection{Shuffler and Jumbler}
\label{sec:sj}

[Body text]

\subsubsection{Placing Threads to Improve Locality}
\label{sec:threadlocality}

[Body text]

\subsubsection{Shuffler}
\label{sec:shuffler}

[Body text]

\subsubsection{Jumbler}
\label{sec:jumbler}

[Body text]

\subsubsection{Shuffler and Jumbler Performance}
\label{sec:sj_performance}

[Body text]

\subsection{FLSCHED for Xeon Phi Manycore Processor}
\label{sec:flsched}

[Body text]

\subsubsection{Lockless Thread Scheduler}
\label{sec:flsched_about}

\cite{Aaronson:2005,Garey:1979,Brun:2008}
~\cite{OM:2008}

\subsubsection{FLSCHED Perfomance}
\label{sec:flsched_performance}

[Body text]

\section{Conclusions}
\label{sec:conclusions}

[Conclusion text]

\section*{Acknowledgments}
\label{sec:acknowledgments}


% The following two commands are all you need in the
% initial runs of your .tex file to
% produce the bibliography for the citations in your paper.
\bibliographystyle{abbrv}
% sample_paper.bib is the name of the BibTex file containing the
% bibliography entries. Note that you *don't* include the .bib ending here.
\bibliography{sample_paper}  
% You must have a proper ".bib" file
%  and remember to run:
% latex bibtex latex latex
% to resolve all references

\end{document}
